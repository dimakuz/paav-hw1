Let $Var^*$ be a finite set of program variables.
Show that $(Var^* \rightarrow \mathbf{Interval},\ \sqsubseteq^\prime)$ where $\forall f_1,f_2 \in Var^* \rightarrow \mathbf{Interval}:(f_1\sqsubseteq^\prime f_2)\longleftrightarrow (\forall v \in Var^*. f_1(v)\sqsubseteq^\prime f_2(v))$ is a complete lattice. 

\subsubsection*{Solution}
We know that $(Var^* \rightarrow \mathbf{Interval},\ \sqsubseteq^\prime)$ is a partial order. To prove that it is a complete lattice we show that every subset has a lowest upper bound and a greatest lower bound. We start from showing a lowest upper bound.
\begin{proof}
	Let $<f_i>$ be a subset of the partial order.
	We use the fact that
	\begin{equation}
	(\mathbf{Interval},\ \sqsubseteq)\ is\ a\ complete\ lattice
	\end{equation}
	Define $f \in Var^* \rightarrow \mathbf{Interval}$ as
	\begin{equation*}
	f(v)=\sqcup <f_i(v)>
	\end{equation*}
	Let $i \in \mathbb{N}$. Let $v \in Var^*$. We know from (6) that $f_i(v)\sqsubseteq \sqcup<f_i(v)>$. So, $f_i(v)\sqsubseteq f(v)$ for every variable $v$. Hence by definition $f_i\sqsubseteq^\prime f$, and this is true for every value of $i$.\\
	
	Let $y$ be an upper bound of $<f_i>$. Let $v \in Var^*$. $\sqcup<f_i(v)>\ \sqsubseteq y(v)$ from (6). So, $f(v)\sqsubseteq y(v)$ and consequently $f\sqsubseteq^\prime y$.\\
	
	$f$ is the greatest lower bound of $(Var^* \rightarrow \mathbf{Interval},\ \sqsubseteq^\prime)$. Similar reasoning applies when showing existance of a lowest upper bound.
\end{proof}
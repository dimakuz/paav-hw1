\begin{enumerate}
\item Prove that the two statements $S_1;\left(S_2;S_3\right)$ and $\left(S_1;S_2\right);S_3$ are semantically equivalent.

\begin{proof}
Let
\begin{enumerate}
	\item $s$, $s^\prime$ states
	\item $S_1$, $S_2$, $S_3$ statements
\end{enumerate}
We start by proving that if

\begin{equation}\label{eq:1}
	\left<\left(S_1;S_2\right);S_3,s\right> \rightarrow s^\prime
\end{equation}

then

\begin{equation}\label{eq:2}
	\left<\left(S_1;S_2\right);S_3,s\right> \rightarrow s^\prime
\end{equation}.

If \refp{eq:1} holds then by $\nsrule{comp}$ exists the derivation tree:

\begin{equation*}
	\inferrule
		{T_1 \\ T_2}
		{\left<\left(S_1;S_2\right);S_3,s\right> \rightarrow s^\prime}_{\nsrule{comp}}
\end{equation*}

where (again, by $\nsrule{comp}$ invocation on $T_1$):
\begin{align*}
	T_1 = & \inferrule
		{T_{11} \\ T_{12}}
		{\left<S_1;S_2,s_0\right>\rightarrow s_1}_{\nsrule{comp}} \\
	T_2 = & \left<S_3,s_2\right>\rightarrow s^\prime
\end{align*}

and

\begin{align*}
	T_{11} = & \left<S_0,s\right>\rightarrow s_1 \\
	T_{12} = & \left<S_1,s_1\right>\rightarrow s_2
\end{align*}

Putting everything together we get:

\begin{equation*}
\inferrule
{
	\inferrule
	{
		\left<S_1,s\right>\rightarrow s_1
		\\
		\left<S_2,s_1\right>\rightarrow s_2
	}
	{
		\left<S_1;S_2,s\right>\rightarrow s_2
	}_{\nsrule{comp}}
	\\
	\left<S_3,s_2\right>\rightarrow s^\prime
}
{
	\left<\left(S_1;S_2\right);S_3,s\right>\rightarrow s^\prime
}_{\nsrule{comp}}
\end{equation*}

The following derivation tree can also be constructed by invocation of $\nsrule{comp}$ twice:

\begin{equation*}
\inferrule
{
	\left<S_1,s\right>\rightarrow s_1 \\
	\inferrule
	{
		\left<S_2,s_1\right>\rightarrow s_2
		\\
		\left<S_3,s_2\right>\rightarrow s^\prime
	}
	{
		\left<S_2;S_3,s_1\right>\rightarrow s^\prime
	}_{\nsrule{comp}}
}
{
	\left<S_1;\left(S_2;S_3\right),s_0\right>\rightarrow s^\prime
}_{\nsrule{comp}}
\end{equation*}

proving that \refp{eq:1}$\Rightarrow$\refp{eq:2}. \\

The other direction is similar, from \refp{eq:2} we can derive the following tree:
\begin{equation*}
\inferrule
{
	\left<S_1,s\right>\rightarrow s_1 \\
	\inferrule
	{
		\left<S_2,s_1\right>\rightarrow s_2
		\\
		\left<S_3,s_2\right>\rightarrow s^\prime
	}
	{
		\left<S_2;S_3,s_1\right>\rightarrow s^\prime
	}_{\nsrule{comp}}
}
{
	\left<S_1;\left(S_2;S_3\right),s_0\right>\rightarrow s^\prime
}_{\nsrule{comp}}
\end{equation*}

and in similar fashion showing that

\begin{equation*}
\inferrule
{
	\left<S_1,s\right>\rightarrow s_1 \\
	\inferrule
	{
		\left<S_2,s_1\right>\rightarrow s_2
		\\
		\left<S_3,s_2\right>\rightarrow s^\prime
	}
	{
		\left<S_2;S_3,s_1\right>\rightarrow s^\prime
	}_{\nsrule{comp}}
}
{
	\left<S_1;\left(S_2;S_3\right),s_0\right>\rightarrow s^\prime
}_{\nsrule{comp}}
\end{equation*}
is a valid derivation tree as well, proving \refp{eq:2}$\Rightarrow$\refp{eq:1}

\end{proof}
\item Construct a statement showing that $S_1;S_2$ is not, in general, semantically equivalent to $S_2;S_1$.

Let $S_1 = x := 1$ and $S_2 = x := 2$, we'll show that $S_1;S_2$ and $S_2;S_1$ are not semantically equivalent. $S_1;S_2$ is derived by:
\begin{equation*}
\inferrule
	{
		\left<x:=1,s\right>\rightarrow s\left[x\mapsto 1\right]_{\nsrule{ass}} \\
		\left<x:=2,s\left[x\mapsto 1\right]\right>\rightarrow s\left[x\mapsto 2\right]_{\nsrule{ass}}
	}
	{\left<x:=1,x:=2,s\right>\rightarrow s\left[x\mapsto 2\right]}_{\nsrule{comp}}
\end{equation*}
$S_2;S_1$ is derived by:
\begin{equation*}
\inferrule
{
	\left<x:=2,s\right>\rightarrow s\left[x\mapsto 2\right]_{\nsrule{ass}} \\
	\left<x:=1,s\left[x\mapsto 2\right]\right>\rightarrow s\left[x\mapsto 1\right]_{\nsrule{ass}}
}
{\left<x:=2,x:=1,s\right>\rightarrow s\left[x\mapsto 1\right]}_{\nsrule{comp}}
\end{equation*}

We get that $\aeval{x}{(s\mapvar{x}{2})} = 2 \ne 1 = \aeval{x}{(s\mapvar{x}{1})}$ thus $s\mapvar{x}{1}\ne s\mapvar{x}{2}$, which shows that the two statements are not semantically equivalent.
\end{enumerate}